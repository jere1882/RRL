
%\newpage{\pagestyle{empty}\cleardoublepage}

\newpage


\begin{center}
\thispagestyle{empty} 
\vspace*{0cm} 

\begin{figure}
\centering%
\epsfig{file=HojaTitulo/EscudoUN,scale=0.7}%
\end{figure}

\textbf{\huge
Búsqueda de Mejoras en la Detección Automática de Estrellas Variables}\\[1.5cm]
\large\textbf{Jeremías Rodríguez} \\ [2.0cm]

Director: Dr. Pablo M. Granitto\\ [0.5cm]

Codirector: Dr. Juan B. Cabral\\ [2.5cm]

Universidad Nacional de Rosario\\
Facultad de Ciencias Exactas, Ingeniería y Agrimensura\\ [1.0cm]
Rosario, Argentina\\
2021\\
\end{center}


\newpage{\pagestyle{empty}\cleardoublepage}
\newpage{\pagestyle{empty}\cleardoublepage}

\newpage
\thispagestyle{empty} \textbf{}\normalsize
\\\\\\%
\textbf{\LARGE Resumen}
\addcontentsline{toc}{chapter}{\numberline{}Resumen}\\\\

La astronomía está atravesando una profunda transformación debido al desarrollo de modernos telescopios terrestres y satelitales, que han fomentado la realización de enormes relevamientos astronómicos. Ante la abrumadora cantidad y calidad de los datos generados, se vuelve imprescindible el uso de procedimientos automatizados. Consecuentemente, diversas técnicas de aprendizaje automatizado y minería de datos surgen como una elección natural a la hora de analizar y extraer información de modernos datasets astronómicos. \\

En este trabajo se hará uso de mediciones generadas por el relevamiento VVV del infrarrojo cercano (realizado en Parnal, Chile), que relevó aproximadamente $10^9$ estrellas durante un período de 5 años. Se aplicarán diversas técnicas de aprendizaje automatizado con el objeto de identificar estrellas de tipo RR Lyrae, las cuales son extremadamente valiosas pues permiten estimar distancias a viejas poblaciones estelares. En concreto, se hará uso de clasificadores de tipo Random Forest y Support Vector Machine, haciendo énfasis en comprender por qué los primeros parecen tener significativamente mejor performance en este tipo de datasets astronómicos.\\


